\documentclass[a4paper,11pt]{article}


\newcommand{\titrecours}{Classifieur de Bayes}
\newcommand{\titretd}{Apprentissage statistique}
\newcommand{\auteur}{Joseph Salmon - St\'ephan Cl\'emen\c con}
\newcommand{\annee}{Master M2MO: 2017/2018}
\newcommand{\ecole}{Universit\'e Paris Diderot}% \titre{}

\usepackage{../sty/td}
\usepackage{../sty/shortcuts_js}
\addbibresource{../biblio/references_all.bib}


\begin{document}
\sloppy
\feuille{1}

\medskip

% \noident
\indent St\'ephan \textsc{Clémençon}  \texttt{<stephan.clemencon@telecom-paristech.fr>}
\\
\indent Joseph \textsc{Salmon} \texttt{<joseph.salmon@telecom-paristech.fr>}

\medskip


% \hrule

\exercice On considère un modèle de classification où le couple aléatoire $(X, Y)$ est de loi $P$ décrite par :
\begin{align*}
\L (X \mid Y= 0) & = \U ([0, \theta]) \\
\L (X \mid Y= 1) & = \U ([0, 1])\\
p & = \PROB\{Y=1\}
\end{align*}
où $p, \theta \in ]0, 1[$ sont fixés. On note $\eta(x)=\PP(Y=1|X=x)$ la fonction de régression correspondante. Donner sa valeur en fonction de $p$ et $\theta$. Donner l'application numérique pour $\theta = 1/2$.

\bigskip

% \hrule

\bigskip

% DGL p.11-12
\exercice  On considère un modèle de classification où le couple aléatoire $(X, Y)$ est de loi $P$ décrite par :
\begin{itemize}
\item la loi de $X$ est une loi de probabilité $P_X$ sur $\RR$
\item la fonction de régression $\displaystyle \eta(x) = \frac{x}{x+\theta}$ où $\theta>0$ fixé.
\end{itemize}
On note $h^*$ le classifieur de Bayes. Expliciter le classifieur de Bayes dans ce modèle. Montrer ensuite que son risque ``0-1'' vaut
\begin{equation*}
	R(h^*)=\int \min(\eta(x),1-\eta(x)) dP_X(x).
\end{equation*}
 Calculer le risque de Bayes lorsque $P_X = \U([0, \alpha\theta])$ où $\alpha>1$.

\bigskip

% \hrule

\bigskip

\exercice On considère un modèle de classification général. On considère le risque de classification pondéré :
\[
L_{\omega}(g) = \EXP \bigl(2\omega(Y) \cdot \ind_{\{Y \neq g(X)\}} \bigr)
\]
où $\omega(0)\geq 0, \omega(1)\geq 0$ et $\omega(0) + \omega(1) =1$.
Donner le classifieur de Bayes et le risque de Bayes pour ce critère.
Quel est l'intérêt de considérer un tel critère ?


\bigskip
%
% \hrule

\bigskip

% DGL p.12-14
\exercice Soit $X = (T, U, V)^T$ où $T, U, V$ sont des variables aléatoires réelles i.i.d. de loi exponentielle $\EE(1)$.
Soit $Y=\ind_{\{T+U+V<\theta\}}$ où $\theta\in\RR_+$ fixé.

\begin{enumerate}
\item Calculer le classifieur de Bayes $g^*(T,U)$, lorsque $V$ n'est pas observée.
 Calculer le risque de Bayes associé à ce classifieur. Donner l'application numérique lorsque $\theta = 9$.

\item On suppose à présent que, seul $T$ est observée. Reprendre les calculs précédents et comparer les risques bayésiens obtenus lorsque  $\theta = 9$.

\item Proposer un classifieur lorsque $X$ n'a aucune composante qui soit observée. Calculer son erreur de classification.
\end{enumerate}


%%%%%%%%%%%%%%%%%%%%%%%%%%%%%%%%%%%%%%%%%%%%%%%%%%%%%%%%%%%%%%%%%%%%%%%%%%%%%%%
%%%%%%%%%%%%%%%%%%%%%%%%%%%%%%%%%%%%%%%%%%%%%%%%%%%%%%%%%%%%%%%%%%%%%%%%%%%%%%%
\section*{Conseils bibliographiques}
\label{sec:bibliographie}
%%%%%%%%%%%%%%%%%%%%%%%%%%%%%%%%%%%%%%%%%%%%%%%%%%%%%%%%%%%%%%%%%%%%%%%%%%%%%%%
%%%%%%%%%%%%%%%%%%%%%%%%%%%%%%%%%%%%%%%%%%%%%%%%%%%%%%%%%%%%%%%%%%%%%%%%%%%%%%%


Vous trouverez ci-dessous quelques points d’entrée utiles pour l'apprentissage automatique:

\begin{itemize}
	\item Théorique et porté sur les aspects probabilistes: \cite{Devroye_Gyorfi_Lugosi96}
	\item Utilitaire et porté sur les aspects pratiques: \cite{Hastie_Tibshirani_Friedman09}
	\item Livre récent porté essentiellement sur l’aspect optimisation: \cite{Shalev-Shwartz_Ben-David14} (et du même auteur sur l'apprentissage en ligne \cite{Shalev-Shwartz11})
	\item Méthodes Bayésiennes et modèles graphiques: \cite{Murphy12}
\end{itemize}



%%%%%%%%%%%%%%%%%%%%%%%%%%%%%%%%%%%%%%%%%%%%%%%%%%%%%%%%%%%%%%%%%%%%%%%%%%%%%%%
\begin{frame}{Bibliographie}
\printbibliography
\end{frame}
 %%%%%%%%%%%%%%%%%%%%%%%%%%%%%%%%%%%%%%%%%%%%%%%%%%%%%%%%%%%%%%%%%%%%%%%%%%%%%%




\end{document}
