%!TEX root = ../main.tex


%%%%%%%%%%%%%%%%%%%%%%%%%%%%%%%%%%%%%%%%%%%%%%%%%%%%%%%%%%%%%%%%%%%%%%%%%%%%%%%
%%%%%%%%%%%%%%%%%%%%%%%%%%%%%%%%%%%%%%%%%%%%%%%%%%%%%%%%%%%%%%%%%%%%%%%%%%%%%%%
\section{Algorithms}
\label{sec:algorithms}
%%%%%%%%%%%%%%%%%%%%%%%%%%%%%%%%%%%%%%%%%%%%%%%%%%%%%%%%%%%%%%%%%%%%%%%%%%%%%%%
%%%%%%%%%%%%%%%%%%%%%%%%%%%%%%%%%%%%%%%%%%%%%%%%%%%%%%%%%%%%%%%%%%%%%%%%%%%%%%%

Considerin the techniques mentioned by \citet{Jaggi13}, you can use a different algorithm a in say \Cref{alg:DC}


\begin{algorithm}
\label{alg:DC}
\caption{DC programming algorithm}
\begin{algorithmic}
\STATE Set $k = 0$ and $\hat{\beta}_0 \in {\rm dom} J_1$, where ${\rm dom} J_1 = \{\beta \in \bbR : J_1(\beta) < \infty\}$
\REPEAT
\STATE Compute $\alpha_{k} \in \partial J_2(\beta_{k})$
\STATE Compute $\hat{\beta}_{k+1} \in \argmin_{\beta \in \bbR^d} J_1(\beta) - \langle \alpha_k, \beta \rangle$
\UNTIL{convergence}
\end{algorithmic}
\end{algorithm}


It is also possible to use the standard citation style \cite{Tibshirani96}.
